\documentclass[RTL]{TAATemplate}
\renewcommand{\rightheadertxt}{احسان آرمیده}
\renewcommand{\leftheadertxt}{مکانیک آماری و ترمودینامیک ۱ / سری اول}

% Packages
\usepackage[headings=runin, 
skip-below={1\baselineskip},
counter-format={qu[1]-}]
{exsheets}

\usepackage{amsmath}

\usepackage{enumitem}

\usepackage{xepersian}

\settextfont[Scale=1]{XB Zar}
\setlatintextfont[Scale=1]{Junicode}
\setdigitfont{XB Zar}

\SetupExSheets[question]{name=}
\SetupExSheets[solution]{name=پاسخ, print=true}
\SetupExSheets[points]{name=نمره, use-name=true}

\def\dbar{{\mathchar'26\mkern-12mu \mathrm{d}}}

\begin{document}

این یک متن آزمایشی است برای آزمودن تمپلیت با متون فارسی و رسم‌الخط آرامی.

از این بابت می‌خواهیم با پکیج ExSheets چند سوال طرح کنیم.

\begin{question}{2}
    صرفاً نشان می‌دهیم که تابع $U = U(T, P)$ دیفرانسیل کاملش چگونه می‌شود؟
\end{question}
\begin{solution}
برای نشان دادن دیفرانسیل کامل تابع مذکور خواهیم داشت:
$$\mathrm{d}U = \left(\frac{\partial U}{\partial T}\right)_{P} \mathrm{d}T + \left(\frac{\partial U}{\partial P}\right)_{T} \mathrm{d}P$$
\end{solution}

\begin{question}{1}
    در این سوال بررسی خواهیم کرد که:
    \begin{enumerate}[label={\alph*)}]
        \item چگونه کار را در یک سیستم ترمودینامیکی هیدرواستاتیکی می‌توان محاسبه نمود؟
        \item چگونه گرما را برای یک گاز کامل محاسبه کنیم؟
    \end{enumerate}
\end{question}
\begin{solution}
    \begin{enumerate}[label={\alph*)}]
        \item $$\dbar W = -P\mathrm{d}V \Rightarrow W  = -\int_{V_i}^{V_f} P\; \mathrm{d}V$$
        \item $$\dbar Q = \mathrm{d}U - \dbar W = C_{V}\mathrm{d}T + P\mathrm{d}V \Rightarrow Q = \int_{T_i}^{T_f} C_{V}\; \mathrm{d}T + \int_{V_i}^{V_f} P\; \mathrm{d}V$$
    \end{enumerate}
\end{solution}

\end{document}
